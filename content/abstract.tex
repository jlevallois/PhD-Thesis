% !TEX root = Clean-Thesis.tex
%
\pdfbookmark[0]{Abstract}{Abstract}
\chapter*{Abstract}
\label{sec:abstract}
\vspace*{-10mm}

For several years, laboratories rely on scanners to capture the world around us,
barely visible to the naked eye, with high accuracy. With the X-ray tomography,
for example, we get a set of data organized on a regular grid, which we call
digital data. Specifically, pixels in 2D and 3D voxel. These raw data need to be
analyzed in order to exploit their fair value.

In this thesis, we focus on extracting differential quantities as the curvature
or the surface normal vectors of the object. These quantities are the basis of
several applications such as surface reconstruction or objects comparison. We
focus on the proof of convergence of these estimators, ensuring the quality of
the results. More precisely, when the resolution of the discretization of the
object increases, we want an estimation more accurate.

Finally, we present a surface classification method for extracting the features,
again in the middle of several applications such as compression of objects. All
these methods are applied to Snow Microstructures from X-ray tomography.

\vspace*{20mm}

{\usekomafont{chapter}Résumé}
\label{sec:abstract-french}
\vspace*{5mm}

Depuis plusieurs années, les laboratoires font appel à des scanners pour
capturer le monde qui nous entoure, difficilement visible à l'œil nu, avec une
grande précision. Avec la tomographie à rayons X par exemple, nous obtenons un
ensemble de données organisées sur une grille régulière, que nous appellerons
données digitales. Concrètement, des pixels en 2D et des voxels en 3D. Ces
données brutes nécessitent d'être analysées afin de pouvoir les exploiter à leur
juste valeur.

Dans cette thèse, nous nous intéressons à extraire des quantités différentielles
comme la courbure ou encore les normales à la surface de l'objet. Ces quantités
sont à la base de plusieurs applications comme la reconstruction de surface ou
encore la comparaison d' objets. Nous nous focalisons sur la preuve de
convergence de ces estimateurs, garantissant ainsi la qualité des résultats.
Plus précisément, lorsque la résolution de la discrétisation de l'objet
augmente, nous voulons des estimations de plus en plus précise.

Enfin, nous présentons une méthode de classification de la surface permettant
d'extraire les zones caractéristiques, encore une fois au cœur de plusieurs
applications comme la compression d'objets. Toutes ces méthodes sont appliquées à
des microstructures de neige issues de tomographes à rayons X.
