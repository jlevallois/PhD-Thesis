% !TEX root = Clean-Thesis.tex
%
% Copyright 2015
% Jérémy Levallois <jeremy.levallois@gmail.com>
%
% This file and related figures are under Creative Commons CC BY-NC-SA 4.0
% See <https://creativecommons.org/licenses/by-nc-sa/4.0/>
%
\pdfbookmark[0]{Abstract}{Abstract}
\selectlanguage{english}
\chapter*{Abstract}
\label{sec:abstract}
\vspace*{-10mm}

3D image acquisition devices are now ubiquitous in many domains of science,
including biomedical imaging, material science, or manufacturing. Most of these
devices (MRI, Backscatter X-ray, micro-tomography, confocal microscopy, PET scans)
produce a set of data organized on a regular grid, which we call digital data,
commonly called pixels in 2D images and voxels in 3D images. Properly processed,
these data approach the geometry of imaged shapes, like organs in biomedical
imagery or objects in engineering.

In this thesis, we are interested in extracting the geometry of such digital
data, and, more precisely, we focus on approaching geometrical differential
quantities such as the curvature of these objects. These quantities are the
critical ingredients of several applications like surface reconstruction or
object recognition, matching or comparison. We focus on the proof of multigrid
convergence of these estimators, which in turn guarantees the quality of
estimations. More precisely, when the resolution of the acquisition device is
increased, our geometric estimates are more accurate. Our method is based on
integral invariants and on digital approximation of volumetric integrals.

Finally, we present a surface classification method, which analyzes digital data
in a multiscale framework and classifies surface elements into three categories:
smooth part, planar part, and singular part (tangent discontinuity). Such
feature detection is used in several geometry pipelines, like mesh compression
or object recognition. The stability to parameters and the robustness to noise
are evaluated with respect to state-of-the-art methods. All our tools for
analyzing digital data are applied to 3D X-ray tomography of snow
microstructures and their relevance is evaluated and discussed.

\vspace*{20mm}

\selectlanguage{french}
{\usekomafont{chapter}Résumé}
\label{sec:abstract-french}
\vspace*{5mm}

Les appareils d'acquisition d'image 3D sont désormais omniprésents dans
plusieurs domaines scientifiques comme l'imagerie biomédicale, la science des
matériaux ou encore l'industrie. La plupart de ces appareils (IRM, scanners à
rayons X, micro-tomographes, microscopes confocaux, TEP scans) produisent un
ensemble de données organisées sur une grille régulière que nous nommerons des
données digitales, ou plus couramment des pixels sur des images 2D et des voxels
sur des images 3D. Lorsqu'elles sont récupérées le plus justement, ces données
approchent la géométrie de la forme capturée (comme des organes en imagerie
biomédicale ou des objets dans l'ingénierie).

Dans cette thèse, nous nous sommes intéressés à l'extraction de la géométrie sur
ces données digitales. Plus précisément, nous nous concentrons à approcher des
quantités géométriques différentielles comme la courbure sur ces objets. Ces
quantités sont les ingrédients critiques de plusieurs applications comme la
reconstruction de surface ou la reconnaissance, la correspondance ou la
comparaison d'objets. Nous nous focalisons également sur les preuves de
convergence asymptotique de ces estimateurs, garantissant en quelque sorte la
qualité de l'estimation. En effet, lorsque la résolution de l'appareil
d'acquisition est augmentée, notre estimation géométrique est plus précise.
Notre méthode est basée sur les invariants par intégration et sur
l'approximation digitale des intégrations volumiques.

Enfin, nous présentons une méthode de classification de la surface, qui analyse
les données digitales dans un système à plusieurs échelles et classifie les
éléments de surface en trois catégories : les parties lisses, les parties
planes, et les parties singulières (discontinuités de la tangente). Ce type de
détection de points caractéristiques est utilisé dans plusieurs algorithmes
géométriques, comme la compression de maillage ou la reconnaissance d'objet. La
stabilité aux paramètres et la robustesse au bruit sont évaluées en fonction des
méthodes de la littérature. Tous nos outils pour l'analyse de données digitales
sont appliqués à des micro-structures de neige provenant d'un tomographe à
rayons X, et leur intérêt est évalué et discuté.
